\documentclass[red]{beamer}
\usetheme{CambridgeUS}
\setbeamertemplate{navigation symbols}{}
\setbeamertemplate{footline}{}
\setbeamertemplate{items}[triangle] 
\renewcommand{\footnoterule}{}

\usepackage[utf8]{inputenc}
\usepackage[slovak]{babel}
\usepackage[IL2]{fontenc}
\usepackage[protrusion=true,expansion=true]{microtype}
\usepackage[none]{hyphenat}

\usepackage{xargs}
\usepackage{graphicx}
\usepackage{wrapfig}
\usepackage[absolute,overlay]{textpos}
\usepackage{amssymb}
\usepackage{amsmath}
\usepackage{verbatim}
\usepackage{eurosym}

\newcommand{\TODO}{{\color{red}!!!!!!!!!!!!!!!!!!!!!!!!}}
\newcommand{\vlnovka}{\raise.35ex\hbox{$\scriptstyle\mathtt{\sim}$}}

\newcommandx{\nadpis}[3][1=1,2]{\uncover<#1-#2>{\textbf{#3}}}
\newcommandx{\odrazka}[3][1=1, 2]{\begin{itemize}\item<#1-#2> #3\end{itemize}}
\newcommandx{\ukazka}[2][1=1,2]{\uncover<#1-#2>{\hspace{1cm}\textit{...demonstration...}}}
\newcommandx{\nadpisOnly}[3][1=1,2]{\only<#1-#2>{\textbf{#3}}}
\newcommandx{\odrazkaOnly}[3][1=1, 2]{\only<#1-#2>{\begin{itemize}\item #3\end{itemize}}}
\newcommandx{\ukazkaOnly}[2][1=1,2]{\only<#1-#2>{\hspace{1cm}\textit{...demonstration...}}}

\begin{document}

\large

\title{3D surface reconstruction using\\ Microsoft Kinect camera}
\subtitle{CESCG 2013}
\author{Matej Kopernický}
\date{12.10.2012}

{
\setbeamertemplate{background canvas}{\includegraphics[width=\paperwidth,height=\paperheight]{title}}
\begin{frame}[plain,t]
\begin{center}
\color{white}
\vspace{1.9cm}
\huge\inserttitle\\
\vspace{0.9cm}
\Large\insertsubtitle\\
\vspace{1.5cm}
\color{black}
\large\insertauthor\\
\bigskip
Comenius University\\
Computer Science, $\mathrm{1^{st}}$ year of master's degree\\
\end{center}
\end{frame}
}



\begin{frame}{Introduction}
\odrazka{Originally my bachelor thesis in 2011/2012}
\pause
\bigskip
\odrazka{First place in ŠVOČ 2012}
\end{frame}



\begin{frame}{3D Reconstruction}
\odrazka{Process of capturing the shape and surface of scanned objects}
\pause
\odrazka{Method using metrological and photometrical information will be presented}
\pause
\centerline{\includegraphics[width=.8\linewidth]{depth_map_photo}}
\end{frame}



\begin{frame}{State of the art}

\begin{columns}
\begin{column}{0.5\textwidth}
\textbf{3D scanners}
\begin{itemize}
\item<2-> big
\item<3-> complex
\item<4-> expensive
\end{itemize}
\bigskip
\nadpis[5]{Alternative?}
\odrazka[6]{Motion sensing input devices}
\end{column}

\begin{column}<1->{0.5\textwidth}
\centerline{\includegraphics[width=.5\linewidth]{scanner}}
\end{column}

\end{columns}
\end{frame}



\begin{frame}{Microsoft Kinect}

\odrazka[1]{Originally an accessory for \textit{Xbox360} gaming console}
\odrazka[2]{Price only \vlnovka\EUR{120}}
\medskip
\nadpis[3]{Sensors}
\begin{itemize}
\item<4-> RGB camera with 640$\times$480 resolution
\item<5-> Infrared sensors with 0.8-4~m range producing 320$\times$240 depth map
\end{itemize}

\begin{columns}
\begin{column}{.5\textwidth}
\uncover<1->{\includegraphics[width=\linewidth]{kinect}}
\end{column}
\begin{column}{.5\textwidth}
\flushright
\uncover<6->{\includegraphics[width=.9\linewidth]{depth_map}}
\end{column}
\end{columns}

\end{frame}



\begin{frame}{Obtaining point cloud}
\odrazka[1]{Distance from camera for each pixel is returned}
\odrazka[2]{Tranformation into orthogonal coordinates is required}
\odrazka[3]{Basic parameters of depth camera have to be known}
\odrazka[4]{Color for each 3D point can be obtained from RGB camera}
\medskip
\centerline{
\uncover<5->{\includegraphics[width=.4\linewidth]{mracno}}
}
\end{frame}



\begin{frame}{Reconstruction}
\odrazka[1]{The goal is to merge a sequence of point clouds into one model}
\odrazka[2]{Knowledge of relative tranformation between captured point clouds is required}
\medskip
\begin{columns}
\begin{column}{.5\textwidth}
\vspace{-1cm}
\odrazka[3]{Scene captured by 6DoF hand-held camera motion}
\odrazka[4]{No SLAM available, camera motion must be estimated}
\end{column}
\begin{column}{.5\textwidth}
\uncover<3->{\hspace{.3cm}\includegraphics[width=.75\linewidth]{6DOF.png}}
\end{column}
\end{columns}
\end{frame}



\begin{frame}{Camera motion estimation}
\odrazka[1]{Estimation can only be calculated from captured depth maps and RGB pictures}
\odrazka[2]{Searching for a relative transformation between all successive frames}
\bigskip
\nadpis[3]{Algorithm outline:}
\begin{itemize}
\item<4->{Find corresponding pairs of points on successive frames}
\item<5->{Estimate transformation}
\item<6->{Iterate until estimation precision is sufficient}
\end{itemize}
\bigskip
\nadpis[7]{How to find corresponding pairs of points?}\\
\nadpis[8]{How to estimate transformation?}
\end{frame}



\begin{frame}{ICP}
\odrazka[1]{ICP algorithm (\textit{Iterative Closest Point})}
\odrazka[2]{Pairing criterion is a distance in 3D space}
\odrazka[3]{Nearest neighbours in two point clouds are established as corresponding pairs}
\begin{columns}
\begin{column}{.4\textwidth}
\vspace{-1.25cm}
\odrazka[4]{Iteratively minimizes distance between point clouds until they "merge"}
\end{column}
\begin{column}{.6\textwidth}
\uncover<5->{\centerline{\includegraphics[width=.8\linewidth]{ICP_steps}}}
\end{column}
\end{columns}
\end{frame}



\begin{frame}{ICP}
\centerline{
\only<1>{\includegraphics[width=.85\linewidth]{kuchyna_rgb1}}
\only<2>{\includegraphics[width=.85\linewidth]{kuchyna_rgb2}}
\only<3>{\Large\textit{...demonstration...}}
\only<4>{\includegraphics[width=.85\linewidth]{kuchyna_result}}
}
\end{frame}



\begin{frame}{ICP}
\odrazka[1]{Distance minimization does not always lead to satisfying results}
\centerline{\uncover<2->{\includegraphics[width=\linewidth]{wall_ICP_all}}}
\bigskip
\begin{textblock*}{4.2cm}(4cm,6.2cm)
\only<3>{\includegraphics[width=\linewidth]{facepalm}}
\end{textblock*}
\odrazka[4]{Danger of falling to local minimum}
\odrazka[5]{Scene geometry is not always sufficient}
\odrazka[6]{Why not take advantage of having RGB pictures?}
\end{frame}



\begin{frame}{SURF}
\odrazka[1]{SURF method (\textit{Speeded Up Robust Features})}
\odrazka[2]{Searches for characteristic points of RGB picture}
\odrazka[3]{Corresponding points of two pictures are paired}
\medskip
\centerline{
\uncover<2->{\alt<2>{\includegraphics[width=\linewidth]{wall_SURF}}{\includegraphics[width=\linewidth]{wall_SURF_lines}}}
}
\end{frame}



\begin{frame}{SURF + ICP}
\odrazka[1]{We can associate RGB characteristic points with 3D points}
\odrazka[2]{New pairing criterion}
\odrazka[3]{Initial transformation estimation before applying ICP}
\medskip
\uncover<4>{\bigskip\centerline{\textit{...demonstration...}}}
\uncover<5->{
\begin{columns}
\begin{column}{.5\textwidth}
\vspace{-1cm}
\odrazka[5]{Helps to avoid local minimum problem}
\begin{textblock*}{2.75cm}(3.1cm,6.5cm)
\only<6>{\includegraphics[width=\linewidth]{picard_happy}}
\end{textblock*}
\odrazka[7]{Speedup - less ICP iterations are needed}
\end{column}
\begin{column}{.5\textwidth}
\centerline{
\uncover<5->{\includegraphics[width=.8\linewidth]{wall_reconstructed_SURF}}
}
\end{column}
\end{columns}
}
\end{frame}



\begin{frame}{Estimating transformation}
\nadpis[1]{How to estimate a transformation using point correspondences?}
\odrazka[2]{Horn's quaternion method\footnote<2->{B.K.P. Horn: \textit{Closed-form solution of absolute orientation using unit quaternions. Journal of the Optical Society of America A, Volume~4, Issue~4, Pages 629-642, 1987.}}}
\bigskip
\nadpis[3]{Quaternions}
\odrazka[4]{''Extension'' of complex numbers by two additional imginary units}
\odrazka[5]{Widely used to represent rotation by 4 real numbers}
\bigskip
\odrazka[6]{Method exploits properties of quaternions}
\odrazka[7]{Minimizes mean squared error of transformation aligning two point clouds}
\end{frame}



\begin{frame}{Nearest neighbour search}
\odrazka[1]{Checking all pairs of points would be very inefficient}
\medskip
\nadpis[2]{How to speed up nearest neighbour search?}
\odrazka[3]{Kd-tree -- space-partitioning data structure}
\odrazka[4]{Subsequently splits space by a hyperplane alternately in all dimensions}
\odrazka[5]{Excludes many points in closest neighbour search}
\centerline{
\uncover<4->{\includegraphics[width=.75\linewidth]{kd-tree}}
}
\end{frame}



\begin{frame}{Implementation}
\odrazka[1]{C\#/.NET4 with Kinect SDK}
\odrazka[2]{Own, optimized implementation of kd-tree and Horn's method}
\odrazka[3]{Used OpenSURF and Math.NET}
\odrazka[4]{User-friendly tools for data recording and reconstruction}
\uncover<5->{\centerline{\includegraphics[width=.75\linewidth]{FreeviewReconstructor}}}
\end{frame}



\begin{frame}{Results}
\centerline{\uncover<1->{\textit{...demonstration...}}}
\bigskip
\centerline{
\uncover<2->{\includegraphics[width=.6\linewidth]{kuchyna}}
}
\end{frame}



\begin{frame}{Results}
\odrazka[1]{Relatively accurate 3D reconstruction of small and medium interiors}
\odrazka[2]{Requires \vlnovka2s to align pair of point clouds}
\odrazka[3]{Can be further optimized (using C++, GPU parallelization...) to run in realtime}
\bigskip\bigskip
\odrazka[4]{Starting point for interior scanning/reconstruction, mid-sized objects modeling, etc.}
\end{frame}



\begin{frame}
\Large{\textbf{\centerline{Thank you for your attention!}}}
\normalsize
%\begin{textblock*}{6cm}(0.5cm,8.5cm)
%\url{s.ics.upjs.sk/~mkopernicky}
%\end{textblock*}
\begin{textblock*}{5cm}(8cm,6.5cm)
\includegraphics[width=\linewidth]{kinect2}
\end{textblock*}
\end{frame}


\end{document}